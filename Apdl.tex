\definecolor{mygreen}{rgb}{0,0.6,0}
\definecolor{mygray}{rgb}{0.5,0.5,0.5}
\definecolor{mymauve}{rgb}{0.58,0,0.82}
\lstset{
basicstyle=\footnotesize,        % the size of the fonts that are used for the code
  breakatwhitespace=false,         % sets if automatic breaks should only happen at whitespace
  breaklines=false,                 % sets automatic line breaking
  captionpos=b,                    % sets the caption-position to bottom
  commentstyle=\color{black},    % comment style
  extendedchars=true,              % lets you use non-ASCII characters; for 8-bits encodings only, does not work with UTF-8
  keepspaces=true,                 % keeps spaces in text, useful for keeping indentation of code (possibly needs columns=flexible)
  keywordstyle=\color{black},       % keyword style
  language=[95]Fortran,                 % the language of the code
  numbers=left,                    % where to put the line-numbers; possible values are (none, left, right)
  numbersep=3pt,                   % how far the line-numbers are from the code
  numberstyle=\tiny\color{gray}, % the style that is used for the line-numbers
  rulecolor=\color{black},         % if not set, the frame-color may be changed on line-breaks within not-black text (e.g. comments (green here))
  showspaces=false,                % show spaces everywhere adding particular underscores; it overrides 'showstringspaces'
  showstringspaces=false,          % underline spaces within strings only
  showtabs=false,                  % show tabs within strings adding particular underscores
  stepnumber=1,                    % the step between two line-numbers. If it's 1, each line will be numbered
  stringstyle=\color{mymauve},     % string literal style
  tabsize=2,                       % sets default tabsize to 2 spaces
  title=\lstname                   % show the filename of files
}
 
\section{APDL Script}
\indent\indent\indent   The APDL script for simulating uniaxial tension in a unit cube is presented below
\begin{lstlisting}
/CLEAR

L = 1

STRETCH =0.2

/PREP7
!*  

ERESX,NO
ET,1,SOLID185
!*  
KEYOPT,1,2,0
KEYOPT,1,3,0
KEYOPT,1,6,0
KEYOPT,1,8,0
!*  

T1 = 1
TB,USER,1,1,9,9  
TBTEMP,T1
TBDATA,,55000e6,9500e6,9500e6,0.33,0.27,0.33,5500e6,3000e6,5500e6
TB,STATE,1,,7

MAT,1
TYPE,1

BLC4,0,0,L,L,L
ESIZE,,L
VMESH,1

NSEL,S,LOC,X,0
D,ALL,UX
NSEL,ALL

NSEL,S,LOC,Y,0
D,ALL,UY
NSEL,ALL

NSEL,S,LOC,Z,0
D,ALL,UZ
NSEL,ALL

/SOL
!*  
ANTYPE,0
NSUBST,100,10000,100
OUTRES,ERASE
OUTRES,ALL,ALL  
OUTRES,SVAR,ALL,,NSVAR
TIME,1

NSEL,S,LOC,X,L
D,ALL,UX,STRETCH
NSEL,ALL

/SOL
/STATUS,SOLU
SOLVE   


FINISH


\end{lstlisting}
